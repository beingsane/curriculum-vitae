\documentclass[a4paper,11pt]{article}

\usepackage{geometry}
\geometry{a4paper,left=2.0cm,right=2.0cm,top=1.5cm,bottom=1.5cm}

\usepackage[english]{babel}
\usepackage[utf8]{inputenc}

\usepackage{hyperref}

\begin{document}
\pagestyle{empty}

\begin{center}
 \huge{\textbf{Curriculum Vitae}}
 \\
 \huge{Álvaro Fernandes de Abreu Justen}
 \\
 \small{(also known as Turicas)}
 \large{\\}
 \large{\today}
 \\
 \small{The latest version of this file is available at:
        \href{http://turicas.info/curriculum}{turicas.info/curriculum}}
\end{center}

\newcommand{\titulo}[1]{\section*{#1}}
\renewcommand{\labelitemi}{$\diamond$}
\renewcommand{\labelitemii}{$\rightarrow$}

\titulo{Contact \& Social Media}
	\begin{itemize}
		\renewcommand{\labelitemi}{}
		\item E-mail \& Google Hangouts: \url{alvarojusten@gmail.com}
		\item Mobile, Signal, Telegram \& WhatsApp: +55 21 9 9898-0141
		\item Skype, GitHub, Twitter, Instagram \& SlideShare: turicas
		\item LinkedIn \& Facebook: alvarojusten
		\item Blog: \href{http://turicas.info/}{turicas.info}
	\end{itemize}

\titulo{About}

Álvaro Justen is free/\textit{libre} software user, collaborator and activist
since 2004; goes to many conferences as speaker, attendee or even organizer;
gives programming and electronics courses and today is a digital nomad: travels
through Brazil (and other countries) disseminating knowledge (specially about
Python, Arduino, free software and open data) while works (remotely) for Onyo
(a Brazilian startup) and as an expert at HackHands. Loves coffee, to cook and
to ride a bike.


\titulo{Professional Experience}
	\begin{itemize}
		\item \href{http://hackhands.com}{\textbf{HackHands}}: works as an
			expert into the platform, helping programmers to solve their
			problems quickly, in real time --- since Februrary 2015.
		\item \href{http://onyo.com}{\textbf{Onyo}}: works as senior backend
			developer, using Python, Django and other cool technologies to
			serve thousands of mobile users --- since September 2014.
		\item \href{http://www.cursodearduino.com.br/}{\textbf{Curso de
			Arduino}}: works as a teacher, giving classes about robotics,
			combining concepts of electronics and programming, using the
			Arduino platform and others --- since January 2011.
		\item \href{http://emap.fgv.br/}{\textbf{Fundação Getúlio Vargas -
			Escola de Matemática Aplicada}}: worked as principal software
			developer (Python), systems administrator and was maintainer of the
			\href{http://pypln.org/}{PyPLN} project (distributed natural
			language processing -- Portuguese Language) --- from November 2011
			to September 2014.
		\item \href{http://www.intelie.com.br/}{\textbf{Intelie}}: worked as
			Python developer, systems administrator and was responsible for
			installations, migrations and operations of the company's
			datacenter monitoring software into some huge customers such as
			Globo.com and iG --- from January 2010 to May 2011.
		\item \textbf{Peta5}: was co-founder of this startup which was focused
			on products to the Brazilian digital TV (SBTVD) market; worked as
			Scrum Master, Python developer (mainly using web2py) and systems
			administrator --- from June 2008 to January 2010.
		\item \href{http://www.metaconsultoria.com/}{\textbf{Meta Consultoria}}
			(junior company -- Engeneering School, Fluminense Federal
			University): was selected as trainee and made the whole training
			course (decided on not joinning the company) --- 2008.
		\item \textbf{\textit{Freelancer}}: worked as Web developer using HTML,
			CSS, JavaScript, PHP and MySQL --- from 2003 to 2008.
	\end{itemize}


\titulo{Skills}
	\begin{itemize}
		\item Natural Languages:
		\begin{itemize}
			\item Portuguese: native, fluent.
			\item English: good reading, writing and talking.
			\item Spanish: good reading, intermediate writing and talking.
		\end{itemize}
	\item Programming Languages:
		\begin{itemize}
			\item Python (10+ years of experience)
			\item Bash (10+ years of experience)
			\item HTML, CSS \& JavaScript (level: intermediate-advanced)
			\item C/C++, PHP, Ruby \& Perl (level: basic)
			\item Erlang \& Elixir (learning now)
		\end{itemize}
		\item Tools (software):
		\begin{itemize}
			\item Git \& GitHub, Mercurial (version control, software
				repository)
			\item Django, web2py \& flask (Python Web frameworks)
			\item Jira (project management)
			\item markdown \& \LaTeX\ (markup language, high-quality document
				preparation system)
			\item Arduino
			\item Debian GNU/Linux
		\end{itemize}
		\item Methodologies \& Other Skills:
		\begin{itemize}
			\item Scrum and eXtreme Programming
			\item Test-driven development
			\item Development and access to REST APIs, Web crawling and parsing
			\item GNU/Linux systems administration
			\item Screencasts
			\item Remote working
			\item Conferences organization
		\end{itemize}
	\end{itemize}


\titulo{Remarkable Achievements}
	\begin{itemize}
		\item \href{http://pythonquito.tk}{PythonQuito} Organizer
			(\textit{Charlas Pythónicas} and \textit{Django Girls}) --- Quito,
			March 2016.
		\item Open data initiatives such as \href{http://brasil.io/}{brasil.io}
			and the \href{https://github.com/turicas/rows}{rows} Python
			library.
		\item Remodeling and development of new versions the
			\href{http://pypln.org/}{PyPLN} project: distributed natural
			language processing (free software) -- from 2013 to 2014.
		\item \href{http://2012.pythonbrasil.org.br/}{PythonBrasil[8]}
			Organizer, the biggest Python conference in Latin America (approx.
			450 attendees from all over the world, 4 days of duration) --- Rio
			de Janeiro, November 2012.
		\item Brazilian reference on Arduino, for a huge number of talks and
			courses on the topic.
		\item Foundation and coordination of the
			\href{https://groups.google.com/forum/#!forum/arduinrio}{ArduInRio}
			user group --- Rio de Janeiro, June 2010.
		\item Foundation and coordination of the
			\href{http://dojorio.org}{Coding Dojo de Niterói} group ---
			Niterói, October 2009.
		\item SeTel (Telecommunications Week) Organizer, Fluminense Federal
			University (approx. 180 attendees) --- Niterói, October 2007.
	 \end{itemize}


\titulo{Academic Education}
	\begin{itemize}
		\item Incomplete graduation in
			\href{http://telecom.uff.br/}{Telecommunications Engeneering} at
			\href{http://www.uff.br/}{Fluminense Federal University} ---
			from 2005 to 2010.
		\item High School: Colégio Ruy Barbosa, Três Rios/RJ.
		\item Elementary School: Escola Nossa Senhora de Fátima, Três Rios/RJ.
	 \end{itemize}


 \titulo{Academic Highlights -- Fluminense Federal University}
	\begin{itemize}
		\item Scholarship on Scientific Research on ReMoTE project (Network
			Monitoring for Power Transmission Lines),
			\href{http://www.midiacom.uff.br/ }{MídiaCom Laboratory} --- from
			December 2008 to January 2010.
		\item Scholarship on Scientific Reserach on
			\href{http://complex.if.uff.br/}{Complex Systems and Statistical
			Physics Group} (many projects), Physics Institute, guided by
			Professor \href{http://profs.if.uff.br/tjpp/}{Thadeu Penna} ---
			from April 2007 to November 2008.
		\item Volunteer Math Teacher, Pre-College Preparatory Course,
			Engineering School --- from March 2007 to December 2007.
		\item Scholarship on \href{http://pet.telecom.uff.br/}{Tutorial
			Education Program on Telecommunications (PET-Tele)}, acting on
			tasks related to research, teaching and university extension to
			enhance the Telecommunications Engineering course, guided by
			Professor \href{http://www.telecom.uff.br/~delavega/}{Alexandre de
			la Vega} --- from September 2006 to March 2008.
		\item Volunteer as beta-tester on the pilot project of University Mesh
			Networks, \href{http://www.midiacom.uff.br/}{MídiaCom Laboratory}
			(GT-Mesh/RNP) --- 2006.
		\item GNU/Linux systems administrator and Physics experiments guide at
			\href{http://www.uff.br/casadadescoberta/}{Casa da Descoberta}
			(Science Dissemination Center) --- July 2005 to September 2006.
	\end{itemize}

\end{document}
